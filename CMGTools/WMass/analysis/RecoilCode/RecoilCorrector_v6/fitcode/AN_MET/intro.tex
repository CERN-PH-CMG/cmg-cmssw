\section{Wmass measurement: goals, strategy and challenges}
\label{sec:WMstrategy}

Generally, precision measurements can be used firstly to confirm the validity of the SM and then, 
once a sufficient level of accuracy has been
achieved, to search for minor deviations from the SM that might indicate new physics
that is explicitly manifest at higher energy scales than have so far been probed. The
motivations for precision measurements of the W boson mass is mainly due to the relation 
of the Higgs, W and top mass in the SM which makes specific predictions for parameter of the electroweak interaction.
As important to any precision measurement as the measured value of the parameter
itself is the uncertainty on that value. Thus it is vital to the W mass analysis to evaluate
possible systematic uncertainties on the fitted mass. 
High precision measurements of the W mass have been made by the experiments at LEP and Tevatron.
The aim of the a measurement at LHC is to have an impact on the world average mW.
{\color{magenta}{Are there systematics that can be lowered in the combination (?)}}.

To measure theW mass we use the following three kinematic variables: the W transverse mass ($m_{T}$), the muon transverse
momentum ($p_{T}^{\mu}$), and the neutrino transverse momentum (MET).
The $m_{T}$ and $p_{T}^{\mu}$ measurements provide a powerful cross-check because of their complementary
systematic uncertainties. The shape of the $m_{T}$ distribution is dominated by the
detector resolution (mainly the resolution due to the recoil system energy measurement),
while the $p_{T}^{\mu}$ spectrum is affected by the transverse momentum of the W boson, and hence
the recoil system and initial-state radiation.
The MET measurement is sensitive to the same systematic uncertainties as both $m_{T}$ and $p_{T}^{\mu}$ 
and has significantly poorer experimental resolution, but this measurement is still useful
for a cross-check. Moreover, since the correlations among these three measurements
are not 100\%, we can combine these results.

The measurement of MW is obtained by a comparison of the spectra of the three different measurement variables with
templates generated from Monte Carlo simulation with varied W masses.  
A binned likelihood comparing collider data and simulated events (a template) is computed
for each of the $m_{T}$, $p_{T}^{\mu}$ and MET distributions. 
Templates are generated from different hypothetical W boson mass values.
Templates are corrected for calibration obtained on the muon momentum scale and for the MET recoil obtained from the
Z~$\rightarrow$~$\mu$~$\mu$ events.

The measured MW is the one of the template that maximizes the agreement with the data.

If we aim at measuring MW with 10-15 MeV of error, we need to be able to control the shape of the distributions at the few per mille level.
{\color{magenta}{Add the plot of the ratio of the MT template with two different mass hyphothesis}}
It’s a precision measurement: many theoretical + experimental aspects need to be sorted out and we need to push to the extreme. 

The measurement focus on the MW measurement for the 7 TeV LHC collision to profit for the low PU scenario.
The measurement is meant to be repeated with the 8TeV and 13 TeV data to lower the statistical errors and systematics (i.e. PDF) errors.

\section{Note content}

This note focus on the work done to calibrate the MET.
The hadronic recoil plays a direct role in the calculation of two of the primary observables in the W mass measurement: the MT and the MET. 
It also figures indirectly in the muon pT measurement through a kinematic requirement on our data the pT(W)~$<$~15 GeV.

In this first chapter, Section~\ref{sec:sample} describe the dataset and MC sample used. In Section~\ref{sec:Sel} we outline the requirements to select the Z and W events.

The second chapter discuss the methods developed to derive and apply the correction to the MET.
In Section~\ref{sec:defRecoil} the definition of the recoil and the decomposition into U1 and U2 is introduced.
In Section~\ref{sec:METDEF} the choice of the MET definition based on the charged tracks only is justified.
In Section~\ref{sec:ParamRecoil} the modelling of the recoil for the Z~$\rightarrow$~$\mu$~$\mu$ is discussed.
In Section~\ref{sec:AppRecoil} the application to the MC of the recoil correction is presented.
In Section~\ref{sec:RecoilValidation} the validation studies of the recoil model are finally discussed.
In Section~\ref{sec:Ztails} the impact of the background to the dimuon selection is investigated.

The third chapter focus on how to port the calibration from the Z to the W.
In the forth chapter we discuss the impact on the Wmass measurement.

\section{References}
\begin{description}
\item CMS AN-2010-332: MIT note
\item CMS AN-2011-459: Veelken, C.
\end{description}


\section{Samples}
\label{sec:sample}

We use data collected by CMS at LHC collisions at 7TeV which correspond to an integrated luminosity of 4.7 $fb^{-1}$.
We selected events with a Z~$\rightarrow$~$\mu$~$\mu$ and W~$\rightarrow$~$\mu$~$\nu$.
Events are triggered with the single muon channel for the W~$\rightarrow$~$\mu$~$\nu$ and Z events.
The data samples were filtered using official JSON files. 

Both data and MC are reconstructed with the CMS software
version $CMSSW\_5\_3\_X$ with Legacy global Tag ($Summer11LegDR-PU\_S13\_START53\_LV6$) for the description of the alignement and calibration conditions.
All generated events are passed through the CMS detector simulation using GEANT4 
and then processed using a reconstruction sequence identical to that used for data.
For the data used in this analysis, there are an average of about 8 reconstructed primary
interaction vertices for each beam crossing. The MC simulation is generated with a different
PU distribution: pile up scenario described by the "S13".

The Z~$\rightarrow$~$\mu$~$\mu$ and W~$\rightarrow$~$\mu$~$\nu$ signals are generated by POWHEG.
The POWHEG generator is chosen since contains the NLO EWK + QCD corrections that are needed to keep low the theoretical systematics on the final Wmass measurement.
Alternative MadGraph samples are available for the Z~$\rightarrow$~$\mu$~$\mu$ and W~$\rightarrow$~$\mu$~$\nu$ sample.
The nominal POWHEG sample is showered with PHYTIA8 while the MADGRAPH smple is showered with PHYTIA6.
Differences in the nominal and alternative V+jets samples arrise also in the tuning "TuneZ2" for the madgraph sample and "4C" for the madgraph sample.

The data are compared to Monte Carlo simulations in the diMuon channel.

Even though the background contributions are expected to be low in the dimuon selection, they can still be
significant in the tails of MET distributions. Background processes considered comprise 
QCD multi–jet, top-antitop pair, V+jets with W~$\rightarrow$~$\tau$~$\nu$ or Z~$\rightarrow$~$\tau$~$\tau$ and and di–bosons (WW, WZ, ZZ) production. 
The $\tau$ lepton decay in the W~$\rightarrow$~$\tau$~$\nu$ and Z~$\rightarrow$~$\tau$~$\tau$  processes is simulated by the TAUOLA MC package.
%QCD multi–jet background events are generated by PYTHIA [7], 
top-antitop and di–boson samples are generated with MadGraph [5]. 
Generated events are normalized to the Z production cross–section XXXXX,
top-antitop events are normalized to the cross–section YYYYY, di–boson events to the NLO cross-section ZZZZZ.


%Minimum bias events generated by PYTHIA are added to all generated Monte Carlo sample
% according to the distribution described in [10]. Simulated events are reweighted in order to
%match the number of pile–up interactions expected in the nominal, the previous and the sub37
%sequent bunch–crossing (“3d” pile–up reweighting [10]) in run periods A and B, respectively.
%The recent TOTEMmeasurement of 73.5 mb [11] is used for the pp inelastic cross–section 1. 
%All generated events are passed through the full Geant [12] based simulation of the CMS apparatus and are reconstructed using release 4 2 3 of the CMS event reconstruction software.

Generated events are normalized to the Z production crosssection XXXXX, top-antitop events are normalized to
the crosssection YYYYY, diboson events to the NLO cross-section ZZZZZ.

In table~\ref{tab:DatasetsData} and table~\ref{tab:DatasetsMC} the data and MC samples used are listed.

\begin{table}[!ht]
\begin{center}
\begin{tabular}{l|r}
\hline
Dataset Name & run range \\
\hline
\hline

  /SingleMu/Run2011A-12Oct2013-v1/AOD & -  \\
  /SingleMu/Run2011B-12Oct2013-v1/AOD & -  \\
  /DoubleMu/Run2011A-ZMu-12Oct2013-v1/RAW-RECO & -  \\
  /DoubleMu/Run2011B-ZMu-12Oct2013-v1/RAW-RECO & -  \\

\hline
\end{tabular}
\caption{Summary of data datasets used.\label{tab:DatasetsData}}
\end{center}
%\end{table}
%\begin{table}[!ht]
\begin{center}
{\footnotesize
\begin{tabular}{l|l|c|c|c|}
\hline
\multicolumn{3}{c}{With Pileup: Processed dataset name is} \\
\multicolumn{3}{c}{(53) Summer11LegDR-PU\_S13\_START53\_LV6-v*/AODSIM} \\
\hline
 Description                     &   Primary Dataset Name   & Tune & PDF & cross-section [pb]\\
\hline
$ DY \rightarrow \ell \ell$  & DYJetsToLL\_M-50\_7TeV-madgraph-pythia6-tauola & Z2 & cteq6L or CT10 &  \\
$ W \rightarrow \ell \nu$ & WJetsToLNu\_TuneZ2\_7TeV-madgraph-tauola & Z2 & cteq6L or CT10 &  \\
$ DY \rightarrow \mu \mu$  &  DYToMuMu\_M-50To250\_ew-BMNNP\_7TeV-powheg-pythia8 & 4C & NNPDF2.3 &  \\
$ W^{+} \rightarrow \mu^{+} \nu$ &  WplusToMuNu\_M-50To250\_ew-BMNNP\_7TeV-powheg-pytha8 & 4C & NNPDF2.3 & \\
$ W^{-} \rightarrow \mu^{-} \nu$ & WminusToMuNu\_M-50To250\_ew-BMNNP\_7TeV-powheg-pytha8 & 4C & NNPDF2.3 & \\
%$ t\bar{t}$ & TTJets_TuneZ2_7TeV-madgraph-tauola &  \\
\hline
\end{tabular}
}
\caption{Summary of Monte Carlo datasets used.
\label{tab:DatasetsMC}}
\end{center}
\end{table}

\section{Selection}
\label{sec:Sel}

The CMS experiment has utilized a particle flow algorithm in event reconstruction. The selection criteria for muon reconstruction and identication are de-
scribed in detail in XXXXX.
The aim of event selection is to :
a) maximise the number of W decay events selected from the available data sample while minimising the fraction
of background events selected (for example recoil cut to suppress top, met cut
to suppress QCD, Z veto, second lepton to reduce the Z and top dilepton);
b) maximimize the data mc agreement to limit the sources of feature that are difficult to simulate (trkmet, eta of the muon);
The Z event selection for the recoil calibration is as close as possible the W event selection to avoid to introduce biases. Each Z decay is reconstructed
from two muons of opposite sign and with an invariant mass window of 10 GeV around the 91 GeV.

The CMS experiment has utilized a particle-flow algorithm in event reconstruction.
The selection criteria for muon reconstruction and identification are described in detail in XXXXX.

The aim of event selection is to :
a) maximise the number of W decay events selected from the available data sample while minimising the fraction of background events selected (for example recoil cut to suppress top, met cut to suppress QCD, Z veto, second lepton to reduce the Z and top dilepton);
b) maximimize the data mc agreement to limit the sources of feature that are difficult to simulate (trkmet, eta of the muon);

The Z event selection for the recoil calibration is as close as possible the W event selection to avoid to introduce biases.
Each Z decay is reconstructed from two muons of opposite sign and with an invariant mass window of 10 GeV around the 91 GeV.